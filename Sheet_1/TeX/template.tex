\documentclass[12pt,A4]{article}
\usepackage[dvipsnames,rgb,dvips]{xcolor}
\usepackage{graphicx}
\usepackage{psfrag}
\usepackage{dcolumn}
\usepackage{bm}
\usepackage{amsmath}
\usepackage{amssymb}
\usepackage[rflt]{floatflt}
\usepackage{latexsym}
\addtolength{\topmargin}{-1.9cm}
\addtolength{\textheight}{5.5cm}
\addtolength{\evensidemargin}{-1.2cm}
\addtolength{\oddsidemargin}{-1.2cm}
\addtolength{\textwidth}{2cm}
\pagestyle{myheadings}
\markright{{\small Jacopo Credi \hfill (910216-T396) \,}}
\begin{document}
\parindent=0cm

\begin{figure}
\caption{\label{fig:1} The figure must have axis labels and tic labels. 
 All symbols, line styles must be explained in the figure or in the caption. This caption must give all information needed to reproduce the figure. Always compare the results of your numerical computations to analytical theory. }
\end{figure}


The format of the solutions must be as follows. On each  sheet there are six questions giving {\bf 1}p each. Separately for each {\bf 1}p-question you must submit one A4 page with 12pt single-spaced text, and with 2cm margins. Each page must  contain  one figure or one table with the corresponding figure or table caption, in addition to the text discussing the results shown in the figure/table.

Every student must hand in her/his own solution on paper. Same rules as for written exams apply: it is not allowed to copy any material from anywhere unless reference is given.

Every answer must contain the following points: brief description of method/algorithm, summary of results, discussion of possible errors and inaccuracies, conclusions. Program code must be appended to web submission, but not to paper submission.
\end{document}
